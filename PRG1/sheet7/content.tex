\section*{7.2}

\subsection*{Programmierkonventionen}
\begin{itemize}
\item Die Zeilenbegrenzung auf 99 Zeichen wurde sehr genau eingehalten.
\item Bei mehrzeilige Docstrings steht zum Teil vor und nach dem \verb+"""+ keine Leerzeile.
\item Nach einem Komma steht kein Leerzeichen (dies ist zwar nicht explizit im Styleguide aber in PEP8 vorgegeben).
\item Es existiert keine \verb+main()+ Funktion und auch keine, wie im Header vorgegeben, Abfrage auf\\
\verb+__name__ == '__main__'+. Die \verb+sixteen_is_dead+ Funktion wird stattdessen direkt am Ende der Datei aufgerufen.
\end{itemize}

\subsection*{Verbesserungsvorschläge}

Die Aufteilung in mehrere Funktionen ist gut gelungen, den Namen \verb+user_interface_2+ könnte man aber z.B. durch \verb+ask_user_which_dice+ ersetzen und dadurch die Aufgabe der Funktion besser wiedergeben.\\

Bei Nutzereingaben wurde \verb+int(input(...))+ verwendet, wenn der Nutzer keine Zahl eingibt, wird dies einen Fehler werfen, der nirgends im Programm aufgefangen wird, dies kann z.B. durch Nutzung von \verb+.isdigit()+ umgangen werden.\\

Die \verb+sixteen_is_dead+ Funktion ist relativ komplex, hier wäre eine weitere Unterteilung in Funktionen sinnvoll. Insbesondere wird das Code-Stück (Zeile 87 und 98)
\begin{lstlisting}[language=Python]
total_points += dice_number
print()
print("Deine aktuelle Punktzahl betraegt:",total_points)
print()
if total_points == 10:
    time.sleep(3)
    continue
else:
    break
\end{lstlisting}
zweimal verwendet.\\

Die Funktionen \verb+set_players+, \verb+setup_number_of_dices+ und \verb+setup_number_of_faces+ sind sich sehr ähnlich, hier könnte man eine allgemeine \verb+setup_numeric_value+ Funktion schreiben, um den dreifachen Code zu reduzieren.\\

Nutzereingaben werden zum Teil in den beiden \verb+user_input+ Funktion geparsed aber auch in der \verb+sixteen_is_dead+ Funktion. Hier würde es sich anbieten, Nutzereingaben nur in den \verb+user_input+ Funktionen zu parsen und z.B. \verb+True+ oder \verb+False+ zurückzugeben, je nach dem, ob der Nutzer noch einmal werfen möchte oder nicht.\\

Der Docstring der Funktion \verb+restart+ ist inkorrekt, die Funktion fragt den Nutzer, ob er das Spiel neustarten will.\\

Es wäre schön, die gewürfelten Zahlen zu sehen und nicht nur die Summe aller Zahlen.\\

Die Dopplung des Nutzerinterfaces ist nicht notwendig, der Nutzer könnte auch bei der ersten Frage \verb+"c"+ eingeben, um den ''Cheating dice'' zu verwenden.\\

\section*{Fazit}
Das Programm ist für Fremde relativ gut zu verstehen, auch wenn die \verb+sixteen_is_dead+ Funktion besser dokumentiert sein könnte.
Das Userinterface ist zwar etwas umständlich aber trotzdem gut gelungen. Leider endet das Programm bei der initialen Abfragen mit einem Fehler, wenn man nichts oder einen String eingibt.